\chapter*{Abstract}
\label{chap:abstract}

This thesis explores the development and evaluation of image registration pipelines for underwater navigation, particularly using \acrfull{fls} data. Developed in collaboration with Blueye Robotics, the thesis focuses on using Fourier-based registration techniques to generate mosaics from sonar frames and estimate odometry for \acrfull{rov}. Two key pipelines are analyzed: the traditional Fourier-Mellin Transform and how it's adapted for sonar registration, and Fourier-based pipeline by \citeauthor{Hurtos2015}, referred to in the text as "Raw Polar". Through testing, it was determined that the Raw Polar Pipeline excels at pure rotations and small translations over the surge axis, whereas the Fourier-Mellin Pipeline is very robust at distant registration regardless of rotation or translation direction. Although it was proposed as a faster alternative to the Fourier-Mellin approach over sonar frames, some problems with the Raw Polar implementation were found that would impact registrations over sway translations and in global alignment. It was also determined that given enough down-sampling, the Fourier-Mellin pipeline might outperform the Raw Polar in total registration time. All in all, the findings suggest that a pipeline running in real time on the drone would enable better navigation while reducing the need for additional positioning tools. 