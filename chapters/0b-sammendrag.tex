\chapter*{Sammendrag}

Denne oppgaven utforsker utvikling og evaluering av bilderegistreringsrørledninger for undervannsnavigasjon, spesielt ved bruk av Forward Looking SONAR (FLS) data. Utviklet i samarbeid med Blueye Robotics, fokuserer oppgaven på å bruke Fourier-baserte registreringsteknikker for å generere mosaikk fra sonarrammer og estimere odometri for Remotely Operated Vehicle (ROV). To nøkkelrørledninger analyseres: den tradisjonelle Fourier-Mellin-transformasjonen og hvordan den er tilpasset for sonarregistrering, og Fourier-basert rørledning av Hurtós et al., referert til i teksten som "Raw Polar". Gjennom testing ble det bestemt at Raw Polar Pipeline utmerker seg ved rene rotasjoner og små translasjoner over surge-aksen, mens Fourier-Mellin Pipeline er svært robust ved fjernregistrering uavhengig av rotasjon eller translasjonsretning. Selv om det ble foreslått som et raskere alternativ til Fourier-Mellin-tilnærmingen over ekkoloddrammer, ble det funnet noen problemer med Raw Polar-implementeringen som ville påvirke registreringer over svaioversettelser og i global justering. Det ble også fastslått at gitt nok ned-sampling, kan Fourier-Mellin-rørledningen overgå Raw Polar i total registreringstid. Alt i alt tyder funnene på at en rørledning som kjører i sanntid på dronen ville muliggjøre bedre navigasjon samtidig som behovet for ytterligere posisjoneringsverktøy reduseres.